\textit{Async} is a syntactic language feature that allows for easy construction of asynchronous non-blocking functions. \textit{Asynchronous} programming lets us write concurrent, not parallel, tasks while looking awfully similar to normal blocking programming. It is a good alternative to \textit{event-driven} programming which tends to be verbose and hard to follow. All \textsc{Async} systems are build around special function that do not return a value but rather a \textit{promise} of a \textit{future} value. When we need the value we tell the program not to continue until the promise is fulfilled. Let's look at the example of downloading 2 files:

\begin{lstlisting}[language=rust, style=boxed, tabsize=2]
async fn get_two_sites_async() {
	// Create two different "futures" which, when run to 
	// completion, will asynchronously download the web pages.
	let future_one = download_async("https://www.foo.com");
	let future_two = download_async("https://www.bar.com");

	// Run both futures to completion at the same time.
	let futures_joined = join!(future_one, future_two);
	// Run them to completion returning their return values
	let (foo, bar) = futures_joined.await;
	some_function_using(foo,bar);
}
\end{lstlisting}

Notice the \texttt{async} keyword in front of the function definition, it means the function will return a promise to complete in the future. The \texttt{join!} statement on line 8 combines the two promises for a future answer to a single promise for two answers. In line 10 we await or 'block' the program until \texttt{futures\_joined} turns into two value. Those can then be used in normal and async functions.

The caller of our \texttt{async} \textit{get\_two\_sites\_async} function will need to be another async function that can await \textit{get\_two\_sites\_async}, or it can be an executor. An executor allows a normal function to await async functions.

Let's go through our example again explaining how this mechanism could work. The syntax and workings of async differ a lot here we will look at the language \textit{Rust}. In rust these promises for a future value are called futures. Until the program reaches line 10 no work on downloading the example sites is done. This is not a problem as the results, \textit{foo} and \textit{bar}, are not used before line 11. The runtime will start out working on downloading \texttt{www.foo.com}, probably by sending out a DNS request. As soon as the DNS request has been sent we need to wait for the answer, we need it to know to which IP to connect to download the site. At this point the runtime will instead of waiting start work on downloading bar where it will run into the same problem. If by now we have received an answer on our DNS request for \textit{www.foo.com} the runtime will continue its work on downloading foo. If not the runtime might continue on some other future available to it that can do work at this point.
