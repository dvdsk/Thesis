Although we did not succeed in building a performant system we found a number of interesting avenues for improvement. These fall into three categories: dynamic scaling, more efficient Group Raft and ranged leases.

\subsection{Dynamic scaling}
The \name design includes load balancing through shaping the systems' configuration to the current load. Using subtree partitioning to spread directories across more or less ministries. In \cref{sec:tradeoff} we discussed the trade-off between read or write performance. This trade-off could be transformed into constrained optimization problem. It would be interesting to have \name{} keep an optimal configuration by continuously solving this problem. To allow users to specify to what degree each file should be replicated the replication factor can be added as additional constraint.

\subsection{Consensus}
The \textit{Group Raft} implementation is very inefficient because it sends only one log entry at the time at fixed intervals. An efficient implementation will make it possible to directly compare \name{} with existing distributed FS. We expect a dramatic speedup by batching Raft log messages and sending them on demand. Since modifications to different directories can never impact each other we can use a combination of \textit{ParallelRaft}~\cite{polarfs} and \textit{Group Raft}.

Currently, any node will start an election if it misses a heartbeat. Ministers can, and sometimes are elected as president. The loss of the minister causes its ex-ministry to pause metadata changes and file writes while a clerk is promoted and a replacement clerk assigned. Banning ministers for taking part in elections and preferring idle nodes to clerks will prevent such pauses.

One reason for re-elections is newly elected presidents taking too long to establishing connections and start emitting heartbeats. A better implementation would re-use the connections a candidate established asking for votes when it becomes president.

\subsection{Leases}
Every write triggers a request to all the clerks to stop giving out read leases. This is inefficient for files experiencing mostly writes or for consecutive small writes. Ministers should keep files locked when the files are experiencing high write load. Ranged writes would then become suitable for concurrently writing a file from multiple clients using small non overlapping chunks.

Ranged locking should enable better performance in workloads with lots of reading and writing. It would be interesting to have this difference quantified.

The locking implementation does not batch requests. We suspect batching will significantly increase throughput under load.
