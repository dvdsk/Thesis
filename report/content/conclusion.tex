We have investigated whether a file system using ministries and with ranged based file locking offers an advantage over existing distributed file systems. While the architecture stays close to the simple Raft consensus algorithm coordinates file locking has become highly complex. The complexity kept us from building a stable and performant implementation. We were able to test the impact of the architecture and ranged locking even with these limitations.

The speed at which we can create files scales almost linearly with the number of ministries. With multiple ministries' directory metadata queries take longer then when using only a single ministry. Mean query performance recovers slightly however as we add more than two ministries. This comes at the cost of increasing tail latencies.

With many clients writing to non overlapping chunks 10\% of a file ranged locking offered substantial performance improvements. Using ranged locking when writing an entire file concurrently from many clients offered no advantage actually slowing down performance compared to locking the entire file.

We got some results that should not be possible given the design. A more reliable implementation is needed to investigate whether these results come from implementation flaws or if the design is flawed. 

% TODO: Check related work for hiearchical raft <09-08-22, dvdsk> 
To enable the ministries' architecture we came up with \textit{Group Raft}. It has proven to be a simple extension with promising scaling characteristics. Ranged file locking enables performance gains in limited circumstances while being easy to implement. Both of these should be studied further. 

We conclude that a distributed file system using multiple small raft clusters offers great scaling characteristics. More research is needed to see if such a system can be made as performant as the current state of the art.
