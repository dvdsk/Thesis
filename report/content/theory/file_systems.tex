\section{File System}
A file system is split into two parts, the files and the directory structure. File properties, or metadata, such as its name, identifier, size etc are stored in the directory. Typically the directory entry itself only contains the file name and its unique identifier. Using the identifier the other metadata for the file can be fetched. The content of the file is split into blocks these blocks are stored on stable storage such as an hard drive or ssd. The file system defines an API to allow modifying the files system providing ways to \textit{create}, \textit{read}, \textit{write}, \textit{seek} and \textit{trunctate} files. 

Usually the system adds a distinction between open and closed files. The apis \textit{read} \textit{write} and \textit{seek} are then only allowed on open files. This makes it possible to provide some concistancy guarentees in a concurrent envirement. For example allowing a file to be opend only if it was not already open. This can prevent a user from corrupting data by writing from multiple processes at the same place in the file. There is no risk to reading the same file from multiple process, even while appending to it from other processes\footnote{The OS can ensure append writes are serialized, this is usefull for writing to al log file where each write call appends an entire log line to a file opend in append mode}. To allow such use a file systems can define opening a file in read-only, append-only or read-write mode. On Linux this is opt in\footnote{see flock or fcntl or mandatory locking}. Even more semantics exist for example allowing opening multiple non overlapping ranges of a file for writing. 

