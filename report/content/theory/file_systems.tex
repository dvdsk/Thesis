\subsection{File System}
A file system is a tool to organise data, the files, usinga a directory. Data properties, or metadata, such as a files name, identifier, size, etc are tracked using the directory. Typically the directory entry only contains the file name and its unique identifier. The identifier allowes the system to fetch the other metadata. The content of the data is split into blocks which are stored on stable storage such as an hard drive or ssd. The file system defines an \ac{api} to operate on it providing methods for \textit{create}, \textit{read}, \textit{write}, \textit{seek} and \textit{trunctate} files. 

Usually a file system adds a distinction between open and closed files. The \acp{api}: \textit{read}, \textit{write} and \textit{seek} can then be restricted to open files. This enables the system to provide some concistancy guarentees. For example allowing a file to be opend only if it was not already open. This can prevent a user from corrupting data by writing concurrently to overlapping ranges in a file. There is no risk to reading from concurrently. Depending on the system reading is even safe while appending concurrently from multiple other processes\footnote{The OS can ensure append writes are serialized, this is usefull for writing to a log file where each write call appends an entire log line}. Enable such guarentees a file systems can define opening a file in read-only, append-only or read-write mode. On Linux these guarentees are opt in\footnote{see \textsl{flock}, \textsl{fcntl} or mandatory locking}. More fine grained semantics exist, such as opening multiple non overlapping ranges of a file for writing. 

