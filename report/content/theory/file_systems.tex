\subsection{File System}
A file system is a tool to organize data, the files, using a directory. Data properties, or metadata, such as a files name, identifier, size, etc. are tracked using the directory interface. Typically, the directory entry only contains the file name and its unique identifier. The identifier allows the system to fetch the other metadata. The content of the data is split into blocks which are stored on stable storage such as a hard drive or SSD. The file system defines a \ac{api} to operate on it, providing methods to \textit{create}, \textit{read}, \textit{write}, \textit{seek} and \textit{truncate} files. 

A file system can add a distinction between open and closed files. The \acp{api}: \textit{read}, \textit{write} and \textit{seek} can then be restricted to open files. This enables the system to provide some consistency guarantees. For example allowing a file to be opened only if it was not already open. This can prevent a user from corrupting data by writing concurrently to overlapping ranges in a file. There is no risk to reading from files concurrently. Depending on the system reading is even safe while appending in parallel from multiple other processes\footnote{The OS can ensure append writes are serialized, this is useful for writing to a log file where each write call appends an entire log line.}. To enable such guarantees a file system can define opening a file in read-only, append-only or read-write mode. On Linux these guarantees are opt-in\footnote{See \textsl{flock}, \textsl{fcntl} or mandatory locking}. More fine-grained semantics exist, such as opening multiple non overlapping ranges of a file for writing. 

