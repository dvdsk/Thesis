The current implementation, while functional, is far from optimal. There are bottlenecks which restrict performance and the system is unstable. It is not feasible for a master’s thesis to build a system as fast and as performant and stable as \ceph{} or \ac{hdfs}. Here I explored whether an architecture using ministries and ranged based file locking offers an advantage to existing solutions. We can still answer that question. 

Before we can design experiments that work around the limitations we need to be clear on what those are:

\begin{itemize}
	\item The \raft{} implementation sends only a single log entry at the time and log entries are sent each heartbeat period instead of whenever data becomes available. Effectively imposing a rate limit on the number of changes made to metadata.
	\item Load balancing only replaces nodes that went down, it does not do runtime subtree partitioning (see:~\cref{sec:subtree}).
	\item When making a large amount of requests changing metadata nodes become unresponsive. This means it starts missing heartbeats which triggers re-elections. % NEEDS to be solved
\end{itemize}

None of these impact range based file locking. We can work around them testing the architecture. To check if the results would change without these limitations I profile the system highlighting any bottlenecks that would impact performance.

\subsection{Ministry Architecture}
Here we test the impact of varying amount of ministries on performance. If more ministries lead to a significant performance increase then the design is a success. 

Given the limitation of unimplemented live subtree partitioning we need to manually configure the file system. Without prior data the load balancer will initialize a single ministry responsible for the root directory. For this test we modify it to initialize n ministries one hosting the root the others the directories 0 till n-1 each located in the root.

If there was no rate limit\footnote{Imposed by the suboptimal \raft{} implementation} then load could increase until the node could no longer keep up. Effectively the node's hardware would impose a (high) rate limit. This assumes that there are no other bottlenecks, possible inherent to the design, in the system. The latter we investigate by profiling the nodes, see:~\cref{sec:profile}.

% TODO actual results

\subsection{Range based file locking}

\subsection{Profiling} \label{sec:profile}
